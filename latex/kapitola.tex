\chapter{Sémantická podobnosť textov} \nopagebreak
\label{kap:clenenie} 
Táto kapitola sa zaoberá konceptom sémantickej podobnosti textov. Preskúmava, ako moderné algoritmy a modely umožňujú analyzovať a porovnávať texty na základe ich významu, nie len na základe použitých slov. Sústreďuje sa na rôzne metódy určovania sémantickej podobnosti, vrátane vektorových reprezentácií textu a transformačných modelov.
\vspace{-2em}
\section{Využitie}
Sémantická podobnosť textov je úloha, ktorá hodnotí mieru, do akej dva textové segmenty vyjadrujú podobný alebo rovnaký význam. Oblasti využitia sú systémy automatického odpovedania, klasifikácia textov, kontrola plagiátorstva \cite{stss}. 

\section{Metódy sémantickej podobnosti textov}
Metódy na meranie sémantickej podobnosti možno rozdeliť do troch hlavných kategórií: štatistické (korpusové) metódy (tab.: \ref{kap:statistical_similarity}), metódy založené na podobnosti reťazcov (tab.: \ref{kap:string_similarity}) a metódy založené na znalostiach (tab.: \ref{kap:knowledge_based_methods}).

\small
\begin{table}[H]
    \centering
    \renewcommand{\arraystretch}{1.2}
    \begin{tabular}{|l|}
        \hline
        \multicolumn{1}{|c|}{\textbf{Štatistická/Korpusová podobnosť}} \\
        \hline
        • \textbf{Latentná sémantická analýza} (LSA) \\
        • \textbf{Generalizovaná latentná sémantická analýza} (GLSA) \\
        • \textbf{Explicitná sémantická analýza} (ESA) \\
        • \textbf{Bodová vzájomná informácia – informačné vyhľadávanie} (PMI-IR) \\
        • \textbf{Normalizovaná Google vzdialenosť} (NGD) \\
        • \textbf{Hyperpriestorový analóg jazyka} (HAL) \\
        \hline
    \end{tabular}
\caption{Štatistické/korpusové metódy podobnosti}
\label{kap:statistical_similarity}
\end{table}

\small
\begin{table}[H]
    \centering
    \renewcommand{\arraystretch}{1.2}
    \begin{tabular}{|l|}
        \hline
        \multicolumn{1}{|c|}{\textbf{Podobnosť reťazcov}} \\
        \hline
        • \textbf{Metriky podobnosti založené na znakoch} \\
        \quad – Longest Common Substring (LCS) \\
        \quad – Damerau-Levenshtein \\
        \quad – Jaro \\
        \quad – Jaro-Winkler \\
        \quad – Needleman-Wunsch \\
        \quad – Smith-Waterman \\
        \quad – n-gram \\
        \quad – syntactic n-gram \\
        • \textbf{Metriky podobnosti založené na termínoch} \\
        \quad – Block Distance \\
        \quad – Kosínusová podobnosť (Cosine Similarity) \\
        \quad – Soft Kosínusová podobnosť (Soft Cosine Similarity) \\
        \quad – Sorensen-Dice Index \\
        \quad – Euklidovská vzdialenosť (Euclidean Distance) \\
        \quad – Jaccardov index \\
        \quad – Jednoduchý zhodný koeficient (Simple Matching Coefficient, SMC) \\
        \quad – Prekrývací koeficient (Overlap Coefficient) \\
        \hline
    \end{tabular}
\caption{Metrika podobnosti reťazcov}
\label{kap:string_similarity}
\end{table}

\begin{table}[H]
    \centering
    \renewcommand{\arraystretch}{1.2}
    \begin{tabular}{|l|}
        \hline
        \multicolumn{1}{|c|}{\textbf{Metódy založené na znalostiach}} \\
        \hline
        • \textbf{Metóda založená na hranách} \\
        \quad – Prístup dĺžky cesty \\
        \quad\quad ◦ Najkratšia dĺžka cesty \\
        \quad\quad ◦ Vážená najkratšia dĺžka cesty \\
        \quad – Relatívne škálovanie hĺbky \\
        \quad – Konceptuálna podobnosť \\
        \quad – Normalizovaná dĺžka cesty \\
        \quad – Leacock \& Chodorow \\
        \quad – Wu a Palmer \\
        • \textbf{Metóda založená na informačnom obsahu} \\
        • \textbf{Metóda založená na vlastnostiach} \\
        • \textbf{Prekrývanie} \\
        \quad – Lesk \\
        \hline
    \end{tabular}
\caption{Metódy podobnosti založené na znalostiach}
\label{kap:knowledge_based_methods}
\end{table}

V zahraničí sa sémantickej podobnosti textov venoval rozsiahly výskum, pričom veľkou zásluhou pre tento výskum mal projekt SemEval. Je to séria medzinárodných seminárov zameraných na spracovanie prirodzeného jazyka. Cieľom ktorých je zlepšenie štádia sémantickej analýzy a vytvorenie kvalitných dátových sád. Každý ročník seminárov obsahuje zbierku úloh, na ktorých sa porovnávajú systémy zameraných na sémantickú analýzu.\footnote{\url{https://semeval.github.io/}} \cite{2015-semeval}

V tomto projekte sa vyskytli úlohy zamerané na sémantickú podobnosť textov od roku 2012 až po rok 2017. Pre zahraničné jazyky existuje mnoho modelov zaoberajúcich sa sémantickou podobnosťou textov. Jedny zo známejších sú BERT modely, pre ktoré existujú verzie vo viacerých jazykoch \cite{dutchbert, alberto, camembert}. Existuje niekoľko zahraničných dátových sád. Dve z nich, s ktorými aj budeme pracovať, sú STS Benchmark a SICK \cite{stsbenchmark1, sick}.


\begin{table}[H]
\centering
\begin{tabular}{|c|p{6cm}|p{6cm}|}
\hline
\textbf{Skóre} & \textbf{English (E)} & \textbf{Slovak (SK)} \\
\hline
5.0 & \multicolumn{2}{|p{12cm}|}{\textit{Dve vety sú ekvivalentné, pretože znamenajú to isté.}} \\
\hline
& The bird is bathing in the sink. & Vták sa kúpe v umývadle. \\
& Birdie is washing itself in the water basin. & Vtáčik sa umýva v umývadle. \\
\hline
4.0 & \multicolumn{2}{|p{12cm}|}{\textit{Dve vety sú skoro ekvivalentné, no líšia sa v nepodstatných detailoch}} \\
\hline
& In May 2010, the troops attempted to invade Kabul. & V máji 2010 sa jednotky pokúsili o inváziu na Kábul. \\
& The US army invaded Kabul on May 7th last year, 2010. & Americká armáda vtrhla do Kábulu 7. mája minulého roku, 2010. \\
\hline
3.0 & \multicolumn{2}{|p{12cm}|}{\textit{Dve vety sú približne ekvivalentné, no niektoré dôležité informácie sa líšia/chýbajú.}}\\
\hline
& John said he is considered a witness but not a suspect. & John povedal, že je považovaný za svedka, ale nie za podozrivého. \\
& "He is not a suspect anymore." John said. & \glqq Už nie je podozrivý,\grqq{} povedal John. \\
\hline
2.0 & \multicolumn{2}{|p{12cm}|}{\textit{Dve vety nie sú ekvivalentné, ale zdielajú spoločné detaily.}} \\
\hline
& They flew out of the nest in groups. & Z hniezda vyleteli v skupinách. \\
& They flew into the nest together. & Spoločne vleteli do hniezda. \\
\hline
1.0 & \multicolumn{2}{|p{12cm}|}{\textit{Dve vety nie sú ekvivalentné, ale týkajú sa tej istej témy.}} \\
\hline
& The woman is playing the violin. & Žena hrá na husliach. \\
& The young lady enjoys listening to the guitar. & Mladá slečna rada počúva gitaru. \\
\hline
0.0 & \multicolumn{2}{|p{12cm}|}{\textit{Dve vety sú úplne odlišné.}}\\
\hline
& John went horseback riding at dawn with a whole group of friends. & John išiel na koni za úsvitu spolu s celou skupinou priateľov. \\
& Sunrise at dawn is a magnificent view to take in if you wake up early enough for it. & Východ slnka za úsvitu je nádherný výhľad, ak sa na to zobudíte dostatočne skoro. \\
\hline
\end{tabular}
\caption{Porovnanie anglických a slovenských viet na základe ich podobnosti}
\label{tab:sentence_comparison}
\end{table}
\vspace{1em}

\noindent
Na slovensku sa výskum sémantickej podobnosti textov postupne vyvíja. Jeden z najvýznamnejších pokrokov pre tento výskum bol vývoj SlovakBERT \cite{slovakbert_uvod2}, ktorý je slovenskou verziou globálne populárneho modelu BERT. Dátové sady v slovenskom jazyku doposiaľ neexistujú, no je možné využívať strojovo preložené dátové sady z iných jazykov.
\vspace{1em}

\section{Cieľ práce}
Cieľom tejto práce je skúmať a analyzovať metódy určovania sémantickej podobnosti textov. Zároveň sa zameriava na porovnanie existujúcich prístupov a zisťovanie naj-presnejšej metódy pre slovenský jazyk. Ďalej sa práca venuje hodnoteniu použiteľnosti týchto metód a analýze ich výsledkov.
